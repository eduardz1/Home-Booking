\documentclass[letterpaper]{report}
\usepackage[utf8]{inputenc}
\usepackage{csvsimple}
\usepackage[hidelinks]{hyperref}
\usepackage{graphicx}
\usepackage{float}
\usepackage{array}
\usepackage{longtable}
\usepackage{makecell}
% \usepackage{multirow}
% \usepackage{array, makecell}
\usepackage{adjustbox}
% \usepackage{showframe}
\usepackage{caption}
\usepackage{color}
\usepackage{listingsutf8}

% \usepackage{pstricks}
% \usepackage{auto-pst-pdf}
% \usepackage{pstricks-add}
% \usepackage{pst-eps}
\newcommand{\tabitem}{~~\llap{\textbullet}~~}

\definecolor{mygreen}{RGB}{154,255,77}
\definecolor{mygray}{gray}{0.5}
\definecolor{myblue}{RGB}{41,141,255}

\lstset{
    language=SQL,
    basicstyle=\small,
    keywordstyle=\color{myblue},
    commentstyle=\color{mygray},
    numberstyle=\color{mygreen},
    numbers=left,
    numberstyle=\tiny\color{mygray},
    breaklines=true,
    breakatwhitespace=true,
    tabsize=2,
    literate={{–}{\textendash}2
              {—}{\textemdash}2},
    inputencoding = utf8/latin1,  % Input encoding
    extendedchars = true,  % Extended ASCII
}

\title{
    \leavevmode{\includegraphics[width=0.8\textwidth]{resources/Universita-degli-studi-di-torino-logo.png}\newline\newline}\\
    Progetto di Piattaforma di Home Booking \\
    \large Laboratorio Basi di Dati 2021/2022
}
\author{Eduard Antonovic Occhipinti, Iman Solaih, Marco Molica}

\begin{document}
\maketitle
\tableofcontents

\chapter{Progettazione Concettuale}
\section{Requisiti Iniziali}
\input{Requisiti_iniziali.tex}

\section{Glossario dei Termini}
\input{Glossario.tex}
\newpage

\section{Requisiti rivisti e strutturati in gruppi di frasi omogenee}
\input{Requisiti_rivisti_e_strutturati_in_gruppi_di_frasi_omogenee.tex}

\newpage
\section{Schema E-R + Business Rules}
\includegraphics[width=\textwidth]{resources/pdf/ER.pdf}

\chapter{Progettazione Logica}
\section{Tavola dei Volumi}
\input{Tavola_volumi.tex}


\section{Tavola delle Operazioni}
\input{Tavola_operazioni.tex}


\section{Ristrutturazione dello schema E-R}
\subsection{Analisi delle ridondanze}
{todo:titoletto}ridondanza 1
All'interno dello schema ER è stata identificata 1 ridondanza: la relazione valutazione tra le entità alloggio e recensione. Questa ridondanza ci permette di ottenere le recensioni effettuate su un alloggio utilizzando solamente le entità alloggio e recensione. Analizziamo la ricerca \bf"Ottenere tutte le recensioni di un alloggio" 
{todo:titoletto}schema con ridondanza
{todo:image_page_5}
{todo:titoletto}schema senza ridondanza
{todo:image_page_6}


% \subsection{Analisi delle Ridondanze}
% \subsection{Eliminazione delle Generalizzazioni}
% \subsection{Partizionamento/Accorpamento di entità e associazioni}
% \subsection{Eventuale scelta degli identificatori principali}

\section{Schema E-R ristrutturato + Business Rules}
\includegraphics[width=\textwidth]{resources/pdf/ER-Ristrutturato.pdf}
\clearpage
\section{Business Rules}
\subsection{Derivate dal testo}

\subsection{Introdotte}
\begin{itemize}
    \item bisnis
    \item rule
    \item rule
    \item rule
    \item bisnis
    \item   rule
    \item   rule
    \item   rule
    \item   rule
    \item       rule
    \item    rule
    \item   rule
    \item   rule
\end{itemize}

\clearpage
\section{Schema Relazionale}
\begin{itemize}
    \item Utente (\underline{Email}, Nome, Cognome, Password, Host, Superhost, Verificato, Carta di identità)
    \item Telefono (\underline{Utente}, \underline{Numero}, Prefisso)
    \item Pagamento (\underline{Utente}, \underline{Numero}, Circuito)
    \item Utente (\underline{Email}, Nome, Cognome, Password, Host, Superhost)
    \item Alloggio (\underline{ID Alloggio}, Nome, Host, Descrizione, Tipologia, Orario check-in, Orario check-out, Costo, Costo pulizia, Posti letto, CAP, Comune, Civico, Via)
          Alloggio (Host) referenzia Utente (Email)
    \item Foto (\underline{Path}, \underline{Alloggio})
          Alloggio (Foto) referenzia Alloggio (ID Alloggio)
    \item Prenotazione (\underline{ID Prenotazione}, Richiedente, ID Alloggio, Data inizio, Data fine, Soggiorno, Stato, Numero ospiti)
          Prenotazione (ID Alloggio) referenzia Alloggio (ID Alloggio)
          Prenotazione (Richiedente) referenzia Utente (Email)
    \item Recensione (\underline{ID Recensione}, Autore, ID Prenotazione, Data, Categoria, Valutazione posizione, Valutazione pulizia, Valutazione qualità-prezzo, Valutazione comunicazione, Corpo)
          Recensione (Autore) referenzia Utente (Email)
          Recensione (ID Prenotazione) referenzia Prenotazione (ID Prenotazione)
    \item Commento (\underline{ ID Commento }, Data, ID Recensione, Autore, Corpo)
          Commento (Autore) referenzia Utente (Email)
          Commento (ID Recensione) referenzia Recensione (ID Recensione)
    \item Lista (\underline{ ID Lista }, Nome, Descrizione, Autore)
          Lista (Autore) referenzia Utente (Email)
    \item Contenuto(\underline{ ID Lista }, \underline{ID Alloggio})
          Contenuto (ID Lista) referenzia Lista (ID Lista)
          Contenuto (ID Alloggio) referenzia Alloggio (ID Alloggio)
\end{itemize}

\chapter{Implementazione}

Riportiamo in seguito alcune query significative per il nostro database

\section{DDL di Creazione dei Database}
\lstinputlisting{../sql/DDL/db_init.sql}

\section{DML di Popolamento di Tutte le Tabelle del Database}
% \subsection{Non so che cosa sta descrivendo crud_data}
\lstinputlisting{../sql/DML/crud_data.sql}

% \subsection{Non so che cosa sta descrivendo dummy_data}
\lstinputlisting{../sql/DML/dummy_data.sql}

\section{Qualche Operazione di cancellazione e modifica}
\lstinputlisting{../sql/DML/operation.sql}

\end{document}