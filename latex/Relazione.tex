\documentclass[12pt, letterpaper]{report}
\usepackage[utf8]{inputenc}
\usepackage{csvsimple}
\usepackage[hidelinks]{hyperref}
\usepackage{graphicx}

\title{
    \leavevmode{\includegraphics[width=0.8\textwidth]{resources/Universita-degli-studi-di-torino-logo.png}\newline\newline}\\
    Progetto di Piattaforma di Home Booking \\
    \large Laboratorio Basi di Dati 2021/2022
}
\author{Eduard Antonovic Occhipinti, Iman Solaih, Marco Molica}

\begin{document}
\maketitle
\tableofcontents

\chapter{Progettazione Concettuale}
\section{Requisiti Iniziali}

\newpage
\section{Glossario dei Termini}
\csvreader[
    tabular = |l|p{4cm}|p{2cm}|p{4cm}|,
    table head = \hline\bfseries Termine & \bfseries Descrizione & \bfseries Sinonimi & \bfseries Collegamenti\\\hline,
    late after line = \\\hline
]{../glossario.csv}{}{
    \csvcoli & \csvcolii & \csvcoliii & \csvcoliv
}

\section{Requisiti rivisti e strtturati in gruppi di frasi omogenee}
\section{Schema E-R + Business Rules}

\chapter{Progettazione Logica}
\section{Tavola dei Volumi}
\section{Tavola delle Operazioni}
\section{Ristrutturazione dello schema E-R}
\subsection{Analisi delle Ridondanze}
\subsection{Eliminazione delle Generalizzazioni}
\subsection{Partizionamento/Accorpamento di entità e associazioni}
\subsection{Eeventuale scelta degli identificatori principali}
\section{Schema E-R ristrutturato + Business Rules}
\section{Schema Relazionale}

\chapter{Implementazione}
\section{DDL di Creazione dei Database}
\section{DML di Popolamento di Tutte le Tabelle del Database}
\section{Qualche Operazione di cancellazione e modifica}

\end{document}