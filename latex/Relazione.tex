\documentclass[12pt, letterpaper]{report}
\usepackage[utf8]{inputenc}
\usepackage{csvsimple}
\usepackage[hidelinks]{hyperref}
\usepackage{graphicx}

\title{
    \leavevmode{\includegraphics[width=0.8\textwidth]{resources/Universita-degli-studi-di-torino-logo.png}\newline\newline}\\
    Progetto di Piattaforma di Home Booking \\
    \large Laboratorio Basi di Dati 2021/2022
}
\author{Eduard Antonovic Occhipinti, Iman Solaih, Marco Molica}

\begin{document}
\maketitle
\tableofcontents

\chapter{Progettazione Concettuale}
\section{Requisiti Iniziali}

\newpage
\section{Glossario dei Termini}
\csvreader[
    tabular = |l|p{4cm}|p{2cm}|p{4cm}|,
    table head = \hline\bfseries Termine & \bfseries Descrizione & \bfseries Sinonimi & \bfseries Collegamenti\\\hline,
    late after line = \\\hline
]{../glossario.csv}{}{
    \csvcoli & \csvcolii & \csvcoliii & \csvcoliv
}

\section{Requisiti rivisti e strtturati in gruppi di frasi omogenee}
Si vuole realizzare una base di dati per un servizio che permette di affittare e prenotare
alloggi di vario tipo ad esempio interi appartamenti, stanze private.
\newline
\newline
Per il dato utente registriamo: indirizzo email, password, nome, cognome, numeri di telefono, carta di identità, metodo di pagamento.
\newline
\newline
Per il dato host registriamo: superhost.
\newline
\newline
Per il dato soggiorno registriamo: data inizio, data fine, idalloggio, idprenotazione.
Ogni utente può avere 0 o più soggiorni.
\newline
\newline
Per il dato alloggio registriamo: nome, indirizzo, comune, descrizione, costo per notte per  persona, costo pulizia, numero di letti, orario check-in, orario check-out, rating medio(?)
Ogni host può possedere uno o più alloggi. Ad ogni alloggio sono associate 0 o più foto. Ogni alloggio offre 0 o più servizi. Per ogni alloggio sono scritte 0 o più recensioni.
\newline
\newline
Gli host per diventare superhost devono soddisfare i  seguenti requisiti:
\begin{itemize}
    \item Devono aver completato almeno 10 soggiorni, per un totale di almeno 100 notti.
    \item Devono aver conservato un tasso di cancellazione dell'1\% 
    (una cancellazione ogni 100 prenotazioni) massimo.
    \item Devono aver mantenuto una valutazione complessiva di 4,8 considerando 
    tutti i soggiorni in tutte le case di sua proprietà
\end{itemize}
Ogni utente può aggiungere 0 o più alloggi tra i preferiti e creare 0 più liste.
Per il dato lista registriamo: descrizione, nome
\newline
\newline
Per il dato prenotazione registriamo: costo prenotazione, idalloggio, stato, numero ospiti, id metodo di pagamento. Ogni prenotazione sarà associata ad un utente ed un host, potrà avere 0 o più ospiti associati. La prenotazione può essere cancellata sia dall'ospite che dall'host: verrà aggiornato lo stato.
\newline
\newline
Al termine del soggiorno, gli ospiti e gli host si possono valutare a vicenda. 
Per il dato recensione registriamo: visibilità, data, idautore, idutente, 
idalloggio, testo, valutazione pulizia, valutazione comunicazione, valutazione 
posizione, valutazione qualità-prezzo.
\newline
\newline
Le recensioni possono essere visibili o non visibili. Diventano visibili quando
entrambi hanno fatto la recensione oppure se uno dei due non ha fatto la recensione, l’altra
diventa visibile dopo 7 giorni dalla fine del soggiorno
\newline
\newline
Per il dato commento registriamo: idautore, testo, idrecensione
Gli host e gli ospiti possono commentare più volte le review in cui sono coinvolti, 
creando un thread di discussione.
\newline
\newline
Il sistema deve supportare le seguenti operazioni:
\begin{itemize}
    \item Una volta a settimana viene effettuato un calcolo per aggiornare 
    il tasso di cancellazione di ciascun host.
    \item Una volta al giorno si controllano le condizioni per la qualifica di 
    superhost e viene aggiornato lo status degli host.
    \item Una volta al mese viene calcolata la classifica degli alloggi più graditi
\end{itemize}

\section{Schema E-R + Business Rules}

\chapter{Progettazione Logica}
\section{Tavola dei Volumi}
\section{Tavola delle Operazioni}
\section{Ristrutturazione dello schema E-R}
\subsection{Analisi delle Ridondanze}
\subsection{Eliminazione delle Generalizzazioni}
\subsection{Partizionamento/Accorpamento di entità e associazioni}
\subsection{Eeventuale scelta degli identificatori principali}
\section{Schema E-R ristrutturato + Business Rules}
\section{Schema Relazionale}

\chapter{Implementazione}
\section{DDL di Creazione dei Database}
\section{DML di Popolamento di Tutte le Tabelle del Database}
\section{Qualche Operazione di cancellazione e modifica}

\end{document}